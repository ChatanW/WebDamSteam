\documentclass[10pt]{beamer}
\usetheme{Warsaw}
\usepackage[T1]{fontenc}
\usepackage[utf8]{inputenc}
\usepackage[frenchb]{babel}
\usepackage{amsmath}
\usepackage{amssymb}
\usepackage{mathrsfs}
\usepackage{amsthm}
\usepackage{listings}
%\usepackage{enumitem}
%\usepackage{tikz}
\usepackage{placeins}
\usepackage{hyperref}

%opening
\title{Adopte un teammate ! }
\author{Maxime Morgado et Anthony Lick}

\begin{document}

\begin{frame}
 \maketitle
\end{frame}

\begin{frame}
\frametitle{Genèse du projet}
\begin{block}{Genèse du projet}
\begin{itemize}
 \item Comment trouver de nouveaux équipiers aimant les mêmes jeux que nous ?
 \item Utiliser le même principe que les sites de rencontre 
 \pause \item Steam : beaucoup de jeux, beaucoup d'utilisateurs, API.
 \pause \item Quels critères utiliser ? 
\end{itemize}
\end{block}
\end{frame}

\begin{frame}
 \frametitle{Steam API}
 
\begin{block}{À quelles données a-t-on accès ?}
  \visible<2-> {
  \begin{itemize}
   \item liste des jeux
   \item liste des amis
   \item temps de jeu 
   \item ...
  \end{itemize}
  }
\end{block}
\begin{block}{À quelles données n'a-t-on pas accès ?}
  \visible<3-> {
  \begin{itemize}
   \item Certaines données propres à chaque jeu (ranking)
   \item Données privées chez certains utilisateurs 
  \end{itemize}
  }
\end{block}
\begin{block}{Implémentation de l'API ?}
  \visible<4-> {
  \begin{itemize}
   \item Python
   \item php
  \end{itemize}
  }
\end{block}
 
\end{frame}

\begin{frame}
 \frametitle{Les critères pour trouver un bon coéquipier}
 
 \begin{block}{But}
 Trouver quelqu'un avec
 \begin{itemize}
  \item les mêmes goûts vidéoludiques
  \item le même niveau sur les jeux en question
  \item des horaires compatibles
 \end{itemize}
 \end{block}
 
 \begin{block}{Idées}
  \begin{itemize}
   \item Regarder les jeux en commun.
   \item Utiliser les systèmes de ranking internes à certains jeux. 
   \item Utiliser le temps de jeu. 
   \item Utiliser les achievements et les badges. 
   \item Utiliser la localisation. 
  \end{itemize}

 \end{block}
 
\end{frame}

\begin{frame}
 \frametitle{Comparaison des temps de jeu}
 
 \begin{block}{But}
  Créer un score de compatibilité à partir des temps de jeu. 
 \end{block}
 
 \only<2-3>{
 \begin{block}{Première idée}
  Filtrage brut des jeux en fonction de l'écart relatif. 
  \only<3> {
  \begin{itemize}
   \item N'a pas beaucoup de sens pour les jeux avec peu d'heures de jeu. 
   \item N'a pas beaucoup de sens pour les jeux avec beaucoup d'heures de jeu. 
   \item Trop binaire
  \end{itemize}
  }
 \end{block}
 }

 \only<4->{
 \begin{block}{Deuxième idée}
  Utiliser un système de paliers pour représenter les différents niveaux des joueurs, en fonction du temps de jeu. 
  \begin{itemize}
  \item[]
  \begin{itemize}
   \item[0] Jeu acheté, mais jamais joué. 
   \item[0-15] Solo \& Multijoueur : Joueur débutant. 
   \item[8-60] Solo : Joueur aguéri. Multijoueur : Joueur débutant.
   \item[40-120] Solo : Mode histoire terminé. Multijoueur : Joueur moyen. 
   \item[90-550] Solo : Modeur. Multijoueur : Bon joueur. 
   \item[500-1200] Multijoueur : Haut niveau. 
   \item[800-$\infty$] Multijoueur : Très haut niveau. 
  \end{itemize}
  \end{itemize}
 \end{block}
 }
 
\end{frame}

\begin{frame}
 \frametitle{Communauté de test} 
 \begin{block}{Communauté de test}
 \begin{itemize}
  \item Récolter une liste d'utilisateurs pour faire des tests. 
  \item Parcours en largeur du graphe des amis Steam (modulo les données privées). 
 \end{itemize}
 Cette liste d'utilisateurs servira aussi à simuler la liste des utilisateurs de notre site. 
 \end{block}

\end{frame}

\begin{frame}
 \frametitle{Création du site Web}
 
 \begin{block}{Interface Web}
  \begin{itemize}
   \item Permet de faire tester notre projet à des amis.
   \item Accès aux profils des coéquipiers trouvés.
   \item Appeler les scripts Python avec PHP. 
   \begin{itemize}
    \item Problèmes d'encodage (PHP et UTF-8). 
   \end{itemize}
   \item Authentification à Steam ? 
   \begin{itemize}
    \item Utilisation d'OpenId. 
   \end{itemize}
  \item Différents tests possibles :
  \begin{itemize}
   \item 3 communautés de taille différentes. 
   \item tester l'algorithme sur sa liste d'amis. 
  \end{itemize}
  \end{itemize}
 \end{block}

\end{frame}

\begin{frame}
 \frametitle{Conclusion}
 
 \begin{block}{Ce qu'il resterait à faire}
  \begin{itemize}
   \item Utiliser d'autres critères : 
   \begin{itemize}
    \item Localisation.
   \end{itemize}
   \item Faire des demandes personnalisées : 
    \begin{itemize}
    \item Création d'un profil.
    \item Ajouts de jeux absents sur Steam.
    \item Choisir les jeux qui nous intéresse le plus.
    \end{itemize}
  \item Réponses plus personnalisées :
  \begin{itemize}
   \item Jeux du palier $0$ en commun.
  \end{itemize}
  \item Amélioration du temps de calcul :
  \begin{itemize}
   \item Précalcul quotidien en faisant des bases de données.
  \end{itemize}

  \end{itemize}
  
 \end{block}
 
 Site Web : 	

\end{frame}


\end{document}
