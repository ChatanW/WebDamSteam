\documentclass[a4paper,10pt]{article}
\usepackage[T1]{fontenc}
\usepackage[utf8]{inputenc}
\usepackage[frenchb]{babel}
\usepackage{amsmath}
\usepackage{amssymb}
\usepackage{mathrsfs}
\usepackage{amsthm}
\usepackage{listings}
%\usepackage{enumitem}
%\usepackage{tikz}
\usepackage{placeins}
\usepackage{hyperref}

%opening
\title{Rapport du projet de Web Data Managment.}
\author{Maxime Morgado et Anthony Lick}

\begin{document}

\maketitle

\section{Présentation du projet}

Notre projet a consisté à réaliser un site de rencontres pour gamers, permettant à des utilisateurs Steam de trouver de nouveaux coéquipiers. 
Pour cela, nous avons utilisé l'API Steam, son implémentation en Python et en PHP. 
Nous avons aussi créé une interface Web, accessible à l'adresse suivante : \url{http://perso.crans.org/morgado/}. 

\section{Présentation des fichiers du projet}

\subsection{Dossier scriptPython}

\subsubsection{Fichiers tests de l'API}

Les fichiers test.py, test2.py, test3.py contiennent les premiers tests concernant l'API. 
En particulier, le fichier test.py contient la fonction qui crée une communauté de test. 

\subsubsection{Nos algorithmes}

Notre première approche, avec un filtrage brut, est implémentée dans score1.py et il est possible de la tester avec test\_projet.py. 
Notre seconde approche, avec les paliers, est implémentée dans score2.py, il est possible de la tester avec test\_projet2.py et test\_projet3.py. 
Les scripts utilisés par notre site Web, basés sur score2.py, sont contenus dans les fichiers script.py et scrip\_friends.py. 
En particulier, le fichier test\_projet2.py peut être utilisé (et modifié) pour tester notre algorithme sans passer par l'interface Web. 
Dans ce cas, on peut obtenir l'ID d'un utilisateur à l'aide du site suivant : \url{https://steamid.io/} (utiliser le champ \textbf{steamID64} fourni par ce site). 

\subsection{Fichiers PHP}

La racine du dossier contient les fichiers PHP implémentant notre site Web. Notamment, le fichier script.php appelle les scripts python et affiche les résultats, 
le fichier texteindex.php permet de se connecter à Steam via OpenId. 

\end{document}
